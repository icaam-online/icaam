%%%%%%%%%%%% Template for ICAAM  %%%%%%%%%%%%%%%%
%%%%%%%%%%%% Please do NOT modify the preamble. %%%%%%%%%%%%%


\documentclass[12pt]{amsart}
%%%%%%%%%%%%%%%%%%%%%%%%%%%%%%%%%%%%%%%%%%%%%%%%%%%%%%%%%%%%%%%%%%%%%%%%%%%%%%%%%%%%%%%%%%%%%%%%%%%%%%%%%%%%%%%%%%%%%%%%%%%%%%%%%%%%%%%%%%%%%%%%%%%%%%%%%%%%%%%%%%%%%%%%%%%%%%%%%%%%%%%%%%%%%%%%%%%%%%%%%%%%%%%%%%%%%%%%%%%%%%%%%%%%%%%%%%%%%%%%%%%%%%%%%%%%
\usepackage{amsfonts}
\usepackage{amssymb}
\usepackage{amsmath}
\usepackage{cite}

\pagestyle{plain} \textwidth140mm \textheight210mm
\oddsidemargin15mm \topmargin15mm
\parskip4pt plus2pt minus2pt
\newtheorem{theorem}{Theorem}[section]
\newtheorem{lemma}[theorem]{Lemma}
\newtheorem{proposition}[theorem]{Proposition}
\newtheorem{corollary}[theorem]{Corollary}
\newtheorem{definition}[theorem]{Definition}
\newtheorem{example}[theorem]{Example}
\newtheorem{remark}[theorem]{Remark}


\begin{document}

%%%%% Enter your information below %%%%%%%%%

\begin{center}
\textbf{{\large {\ Stability of basis property of a type of problems \\[0pt]
with nonlocal perturbation of boundary conditions}}}\\[0pt]
\medskip Nurlan Imanbaev$^{1}$, Makhmud Sadybekov$^{2}$\\[0pt]
\medskip \textit{\ $^{1}$ Institute of Mathematics and Mathematical
Modeling, Kazakhstan, and }

\textit{South Kazakhstan State Pedagogical Institute, Kazakhstan}

\textit{imanbaevnur@mail.ru\\[0pt]
$^{2}$ Institute of Mathematics and Mathematical Modeling, Kazakhstan}

\textit{sadybekov@math.kz\\[0pt]
}
\end{center}

\vspace{1cm} \textbf{Abstract:} This report is devoted to a spectral problem
for a multiple differentiation operator with an integral perturbation of
boundary conditions of one type which are regular, but not strongly regular:
\begin{equation}
l(u)\equiv -u^{\prime \prime }(x)=\lambda u(x),~0<x<1,  \label{1}
\end{equation}%
\begin{equation}
U_{1}(u)\equiv u^{\prime }(0)-u_{0}^{\prime 1}\overline{p(x)}u(x)dx,~p(x)\in
L_{2}(0,1),  \label{2}
\end{equation}%
\begin{equation}
U_{2}(u)\equiv u(0)-u(1)=0.  \label{3}
\end{equation}%
Here $\alpha \neq 0$ is an arbitrary complex number.

The unperturbed problem ($p(x)\equiv 0$) has an asymptotically simple
spectrum, and its system of normalized eigenfunctions creates the Riesz
basis. We construct the characteristic determinant of the spectral problem
with an integral perturbation of the boundary conditions. The perturbed
problem can have any finite number of multiple eigenvalues. Therefore, its
root subspaces consist of its eigen and (maybe) adjoint functions. It is
shown that the Riesz basis property of a system of eigen and adjoint
functions is stable with respect to integral perturbations of the boundary
condition.

Throughout this note we mainly use techniques from our works [1].

\medskip \noindent \textbf{Keywords:} Riesz basis, regular boundary
conditions, eigenvalues, root functions, spectral problem, integral
perturbation of boundary condition, characteristic determinant

\medskip \noindent \textbf{2020 Mathematics Subject Classification: }{35J05,
35J08, 35J25 }

% For MSC 2020, see https://zbmath.org/static/msc2020.pdf

\begin{thebibliography}{9}
\bibitem{1} M.A. Sadybekov, N.S. Imanbaev, On the basis property of root
functions of a periodic problem with an integral perturbation of the
boundary condition, Differential Equations, vol. 48, no 6,  896--900, 2012.
\end{thebibliography}

\end{document}
